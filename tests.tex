\section{Implementation tests}

\begin{figure}
    \centering
    \begin{subfigure}[b]{0.5\textwidth}
        \resizebox{1.0\textwidth}{!}{\input{plots/access_time_0}}
        \caption{3D Gaussian}
        \label{fig:access_time_0}
    \end{subfigure}\hfill
    \begin{subfigure}[b]{0.5\textwidth}
        \resizebox{1.0\textwidth}{!}{\input{plots/access_time_1}}
        \caption{Squared Distance To Line}
        \label{fig:access_time_1}
    \end{subfigure}
    \begin{subfigure}[b]{0.5\textwidth}
        \resizebox{1.0\textwidth}{!}{\input{plots/access_time_2}}
        \caption{Linear Distance From 3 Points}
        \label{fig:access_time_2}
    \end{subfigure}\hfill
    \begin{subfigure}[b]{0.5\textwidth}
        \resizebox{1.0\textwidth}{!}{\input{plots/access_time_3}}
        \caption{Schwefel Function}
        \label{fig:access_time_3}
    \end{subfigure}
    \caption{Plots of mean access time for different functions}
    \label{fig:access_time}
\end{figure}

\begin{figure}
    \centering
    \begin{subfigure}[b]{0.5\textwidth}
        \resizebox{1.0\textwidth}{!}{\input{plots/balance_mean_0}}
        \caption{3D Gaussian}
        \label{fig:balance_mean_0}
    \end{subfigure}\hfill
    \begin{subfigure}[b]{0.5\textwidth}
        \resizebox{1.0\textwidth}{!}{\input{plots/balance_mean_1}}
        \caption{Squared Distance To Line}
        \label{fig:balance_mean_1}
    \end{subfigure}
    \begin{subfigure}[b]{0.5\textwidth}
        \resizebox{1.0\textwidth}{!}{\input{plots/balance_mean_2}}
        \caption{Linear Distance From 3 Points}
        \label{fig:balance_mean_2}
    \end{subfigure}\hfill
    \begin{subfigure}[b]{0.5\textwidth}
        \resizebox{1.0\textwidth}{!}{\input{plots/balance_mean_3}}
        \caption{Schwefel Function}
        \label{fig:balance_mean_3}
    \end{subfigure}
    \caption{Plots of mean tree balance factors for different functions}
    \label{fig:balance_mean}
\end{figure}

\begin{figure}
    \centering
    \begin{subfigure}[b]{0.5\textwidth}
        \resizebox{1.0\textwidth}{!}{\input{plots/nodes_count_0}}
        \caption{3D Gaussian}
        \label{fig:nodes_count_0}
    \end{subfigure}\hfill
    \begin{subfigure}[b]{0.5\textwidth}
        \resizebox{1.0\textwidth}{!}{\input{plots/nodes_count_1}}
        \caption{Squared Distance To Line}
        \label{fig:nodes_count_1}
    \end{subfigure}
    \begin{subfigure}[b]{0.5\textwidth}
        \resizebox{1.0\textwidth}{!}{\input{plots/nodes_count_2}}
        \caption{Linear Distance From 3 Points}
        \label{fig:nodes_count_2}
    \end{subfigure}\hfill
    \begin{subfigure}[b]{0.5\textwidth}
        \resizebox{1.0\textwidth}{!}{\input{plots/nodes_count_3}}
        \caption{Schwefel Function}
        \label{fig:nodes_count_3}
    \end{subfigure}
    \caption{Plots of nodes count for different functions}
    \label{fig:nodes_count}
\end{figure}

\begin{figure}
    \centering
    \begin{subfigure}[b]{0.5\textwidth}
        \resizebox{1.0\textwidth}{!}{\input{plots/error_0}}
        \caption{3D Gaussian}
        \label{fig:error_0}
    \end{subfigure}\hfill
    \begin{subfigure}[b]{0.5\textwidth}
        \resizebox{1.0\textwidth}{!}{\input{plots/error_1}}
        \caption{Squared Distance To Line}
        \label{fig:error_1}
    \end{subfigure}
    \begin{subfigure}[b]{0.5\textwidth}
        \resizebox{1.0\textwidth}{!}{\input{plots/error_2}}
        \caption{Linear Distance From 3 Points}
        \label{fig:error_2}
    \end{subfigure}\hfill
    \begin{subfigure}[b]{0.5\textwidth}
        \resizebox{1.0\textwidth}{!}{\input{plots/error_3}}
        \caption{Schwefel Function}
        \label{fig:error_3}
    \end{subfigure}
    \caption{Plots of mean and maximum errors for different functions}
    \label{fig:error}
\end{figure}

\begin{figure}
    \centering
    \begin{subfigure}[b]{0.5\textwidth}
        \resizebox{1.0\textwidth}{!}{\input{plots/build_time_0}}
        \caption{3D Gaussian}
        \label{fig:build_time_0}
    \end{subfigure}\hfill
    \begin{subfigure}[b]{0.5\textwidth}
        \resizebox{1.0\textwidth}{!}{\input{plots/build_time_1}}
        \caption{Squared Distance To Line}
        \label{fig:build_time_1}
    \end{subfigure}
    \begin{subfigure}[b]{0.5\textwidth}
        \resizebox{1.0\textwidth}{!}{\input{plots/build_time_2}}
        \caption{Linear Distance From 3 Points}
        \label{fig:build_time_2}
    \end{subfigure}\hfill
    \begin{subfigure}[b]{0.5\textwidth}
        \resizebox{1.0\textwidth}{!}{\input{plots/build_time_3}}
        \caption{Schwefel Function}
        \label{fig:build_time_3}
    \end{subfigure}
    \caption{Plots of build times for different functions}
    \label{fig:build_time}
\end{figure}

The kd-tree build was tested in the domain of $\mathbf{D} = [-1,\,1] \times [-1,\,1] \times [-1,\,1]$.
The tests were designed to measure the quality of the created tree as well as performance of its creation.

\subsection{Test functions}

The tests were performed for 4 different test functions:
\begin{enumerate}
    \item \textbf{3DGaussian} $f(x,y,z) = e^{\frac{x^2+y^2+z^2}{2 \cdot 0.1^2}}$,
    \item \textbf{Squared Distance To Line} the reference
        line passes through points $(-1,-1,-0.5)$ and $(0.5, 1, 0)$,
    \item \textbf{Linear Distance From 3 Points} is a mean of distances from points
        $(0.5, 0.5, 0.5)$, $(0.5, 0.5, -0.5)$ and $(0,0,0)$,
    \item \textbf{Schwefel Function} $f(x,y,z) = \sum\limits_{x_i \in \{ x, y, z \} } { - x_i \sin{\sqrt{500 |x_i|}}}$ 
        \footnote{\url{http://www.geatbx.com/docu/fcnindex-01.html\#P150_6749}}
\end{enumerate}

All of the functions were assumed to be analysed only in the domain of $\mathbf{D}$.

\subsection{Test design}

The building process was analysed for each of the defined function.
Each function was tested with a number of accuracies,
i.e. $\{1.0,\, 0.9,\, 0.8,\, 0.7,\, 0.6,\, 0.5,\, 0.4,\, 0.3,\, 0.2,\, 0.1\}$,
meaning the maximal difference between kd-tree's output value and the actual
output of the function (error).
Both methods of splitting the boxes were tested, \textbf{half\_box\_splitter}
and \textbf{gradient\_box\_splitter}.

For each case, following quantities were measured.
\begin{itemize}
    \item \textbf{average access time} a uniform grid of points in the
        $\mathbf{D}$ is created and for each point and access is performed, the
        procedure is repeated 100 times for more meaningful results, see Figure~\ref{fig:access_time}
    \item \textbf{balance factor} depths of all the leaves are collected, mean
        value and standard deviation of the result is returned. The quotient of
        these values gives an appropriate measurement of the tree's
        unbalancing, see Figure~\ref{fig:balance_mean}
    \item \textbf{node count} number of nodes in the kd-tree, see Figure~\ref{fig:nodes_count}
    \item \textbf{error} mean and maximum value of difference between the
        kd-tree's approximation and the actual value of the function, see Figure~\ref{fig:error}
    \item \textbf{build time}, see Figure~\ref{fig:build_time}
\end{itemize}


